\chapter{Zaključak i budući rad}
		
		
	 		{Zadatak naše grupe bio je napraviti web aplikaciju za ubrzano 	javljanje informacija o nestalim osobama te javljanje spasilaca na akcije spašavanja istih. Nakon dva ciklusa nastave i praznika, ostvarili smo cilj koji nam je zadan. Rad na aplikaciji razdijeljen je bio na tri faze.\\
	 			Prva faza sastojala se od okupljanja članova tima koji će razvijati aplikaciju, dodjelu projektnog zadatka te rada na dokumentiranju zahtjeva i osmišljaju najboljeg tijeka rada.\\
	 			Druga faza rada na aplikaciji bila je rezervirana za izradu osnovnih funkcija to jest za generalnu funkcionalnost aplikacije (prijava korisnika u sustav). Za vrijeme izrade generalne funkcionalnosti došlo je do nekoliko problema koji su rezultirali mijenjanjem do tada korištene tehnologije na novu,uvod JPA u projekt, kako bismo mogli lakše manipulirati s podatcima u bazi podataka. Prije prve predaje osnovne inačice aplikacije također su izrađeni obrasci i dijagrami(obrasci uporabe,sekvencijski dijagrami, model baze podataka i dijagrami razreda). Izrada ovih idejnih dijelova dokumentacije kasnije je služila \textit{frontend} i \textit{backend} timovima kao osnova za razvoj daljnjih funkcija aplikacije.\\
	 			Treća faza realizacije aplikacije bila je najintenzivniji dio rada na aplikaciji. Intenzitet je pristizao od neiskustva članova timova s novim tehnologijama korištenima u projektu. Timovi su bili zaduženi za izradu ostalih dijelova zahtjeva za projekt, točnije dijelova aplikacije za razne vrste korisnika kao i samog admina aplikacije. Uz rad na funkcionalnostima također su napravljeni i ostali UML dijagrami i ostatak dokumentacije koja će služiti budućim korisnicima kao naputak za korištenje i snalaženje u slučaju preinake sustava. Izvrsno izrađen kostur aplikacije služio je \textit{backend} i \textit{frontend} timovima kao ubrzanje rada zato što je lakše bilo izbjegavati eventualne pogreške koje bi koštale sati rada na aplikaciji bez uroda.\\ 
	 			Komunikacija između članova većinski je rađeno preko WhatsApp aplikacije te timskog rada na Discord-u. Moguć daljnji napredak aplikacije značio bi izrada mobilne aplikacije koja bi značila ostvarenje i više nego što je bio cilj.\\
	 			Sudjelovanje u izradi web aplikacije bilo je vrijedno iskustvo svim članovima tima zbog intenziteta rada, te učenja novih tehnologija. Uz rad i nove tehnologije tim je također naučio i važnost dobre koordinacije članova time kao i vremenske organiziranosti, koja je značila izradu projekta u predviđenom vremenu. Naš tim je iznimno zadovoljan dostignutim ciljem usprkos moguće dorade i usavršavanja samog projekta bez obzira na minimalno ili nikakvo iskustvo u ovakvim projektima i tehnologija.
	 	 } 
		
		\eject 